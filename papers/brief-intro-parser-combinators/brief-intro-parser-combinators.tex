\documentclass{jarticle}
\usepackage{url}
\usepackage{listings}
\usepackage{../sty/epic}
\usepackage{../sty/eepic}
\usepackage{../sty/ecltree}

\title{文脈自由文法とパーサコンビネータ}
\author{Pasberth}
\date{December 20, 2013}

\begin{document}

\maketitle

\begin{abstract}
この文書は文脈自由文法とパーサコンビネータの簡単な教科書である.
これを読めば,きっと,文脈自由文法とパーサコンビネータを理解する手助けとなる
であろう.本書は主にHaskellにおけるパーサコンビネータを扱う.
たとえば,ParsecやAttoparsec,Trifectaの使い方を簡単に勉強するのが本書の
目標である.もしあなたがパーサコンビネータに興味があるなら,ぜひ手に
とってみてほしい.きっと役に立つだろう.
\end{abstract}

\tableofcontents

\section{はじめに}

\subsection{ソースコード}

この文書はオープンソースである.
完全なソースコードは,2013年12月20日現在,
\url{https://github.com/pasberth/pasberth.github.io/tree/default/papers/brief-intro-parser-combinators}で
公開されている.この文書自体のソースコードや,例に使用されるソースコードが
そこから入手できる.

\subsection{ソースコードの入手方法}

もっとも簡単な方法はgitを使用することである.
次のようなシェルコマンドで簡単に入手できる.


\lstset{basicstyle=\ttfamily,}
\begin{lstlisting}
git clone https://github.com/pasberth/pasberth.github.io.git
cd pasberth.github.io
git checkout default
cd papers/brief-intro-parser-combinators
\end{lstlisting}

\section{文脈自由文法}

言語の規定のために用いる概念である文脈自由文法({\it context-free grammer})
を導入する.文脈自由文法は略して文法({\it grammer})ということもある.

文法は,たいていの形式言語の構成要素がもつ階層構造を,自然な形で
表現したものである.たとえば,Javaのif-else文は,
\begin{center}
{\bf if} ( 式 ) 文 {\bf else} 文
\end{center}
である.すなわち,if-else文は,キーワード{\bf if},左かっこ,式,右かっこ,
文,キーワード{\bf else},そして1つの文を並べたものである.
式をあらわすための変数を{\it expr},文をあらわすための変数を{\it stmt}とすると,
この構造化規則は,次のように表現できる.
\begin{center}
{\it stmt} $\longrightarrow$ {\bf if} ( {\it expr} ) {\it stmt} {\bf else} {\it stmt}
\end{center}
このような規則を{\bf 生成規則}とよぶ.
キーワード{\bf if}や``("のような字句表現を{\bf 終端記号}とよぶ.
{\it expr}や{\it stmt}のような変数を{\bf 非終端記号}とよぶ.

\subsection{文法の定義}

文法は次の4つの要素からなる.
\begin{itemize}
\item 終端記号({\it terminal symbol})の集合.
\item 非終端記号({\it nonterminal symbol})の集合.
\item 生成規則({\it production})の集合.
\item 開始記号としての,1つの非終端記号の指定.
\end{itemize}
文法は,生成規則を列挙することによってさだめられる.その際,開始記号に対する
生成規則を先頭におくようにする.{\bf if}や{\bf else}のように,太字であらわされた
記号は終端記号である.``("といった符号は終端記号である.
{\it expr}や{\it stmt}のようにイタリック体であらわされた
記号は非終端記号である.生成規則を,
\begin{center}
$\alpha \longrightarrow \beta_1 \beta_2 \ldots \beta_n$
\end{center}
のように書き,``$\alpha$は$\beta_1 \beta_2 \ldots \beta_n$という形式をもつことができる"と読む.
$\alpha$はその生成規則の{\bf 頭部}または{\bf 左辺}とよばれる.
$\beta_1 \beta_2 \ldots \beta_n$はその生成規則の{\bf 本体}または{\bf 右辺}とよばれる.

文脈自由文法でS式を定義する簡単な例を述べる.
まず,自然言語でS式を定義する.
\begin{enumerate}
\item 加算個の変数 $x_1,x_2,\ldots,x_n$ はS式である.
\item {\sl M}と{\sl N}がS式ならば,({\sl M}.{\sl N})はS式である.
\end{enumerate}
この定義から,たとえば,($x_1$.$x_2$)はS式であるとわかる.
これを文脈自由文法を用いて記述する.
\begin{enumerate}
\item {\it variable} $\longrightarrow$ $x_1$\\
      {\it variable} $\longrightarrow$ $x_2$\\
      :\\
      {\it variable} $\longrightarrow$ $x_n$
\item {\it s-expression} $\longrightarrow$ {\it variable}
\item {\it s-expression} $\longrightarrow$ ( {\it s-expression} . {\it s-expression} )
\end{enumerate}

\subsection{導出}

文法は,開始記号から出発して,非終端記号をそれに対する生成規則の本体で
おきかえる,という操作の繰り返しによって,記号列を導出する.開始記号から導出できる
ことのできる終端記号列が,その文法によって定義される言語を形成する.

たとえば, ($x_1$.$x_2$) が {\it s-expression} であることは,次のようにして説明できる.

\begin{enumerate}
\item 生成規則(1)から $x_1$ は {\it variable} なので,
      生成規則(2)から $x_1$ は {\it s-expression} である.
\item 生成規則(1)から  $x_2$ は {\it variable} なので,
      生成規則(2)から $x_2$ は {\it s-expression} である.
\item $x_1$ と $x_2$ は {\it s-expression} なので,
      生成規則(3) から ($x_1$.$x_2$) は {\it s-expression} である.
\end{enumerate}

\subsection{解析木}

解析木({\it parse tree})は,言語に含まれる文字列が,その文法の開始記号から,
どのように導出できるかを図式したものである.解析木のことを,構文木({\it syntax tree})と
いうこともある.解析木を形式的に定義することもできるが,
ここでは例を見て理解してほしい.次の図は, ($x_1$.$x_2$) の解析木である.
\begin{center}
\GapDepth=20pt
\GapWidth=40pt
\begin{bundle}{{\it s-expression}}
\chunk{(}
\chunk
{
\begin{bundle}{{\it s-expression}}
\chunk
{
\begin{bundle}{{\it variable}}
\chunk{$x_1$}
\end{bundle}
}
\end{bundle}
}
\chunk{.}
\chunk
{
\begin{bundle}{{\it s-expression}}
\chunk
{
\begin{bundle}{{\it variable}}
\chunk{$x_2$}
\end{bundle}
}
\end{bundle}
}
\chunk{)}
\end{bundle}
\end{center}

\subsection{抽象構文木}

抽象構文木({\it abstract syntax tree})は,解析木からよけいな文字を取り除いたものである.
たとえば,($x_1$.$x_2$)の解析木に``("や``."や``)"といった文字が含まれるが,情報として
必要とは思えない.なぜならば,これがあったとしてもなかったとしても,それが定義(3)のものだと
はっきりわかるからである.しかし,$x_1$や$x_2$は情報として必要である.``("や``."や``)"と
いった文字を取り除き,$x_1$や$x_2$のみを含む木を抽象構文木という.ここでは例をみて理解して
ほしい.
\begin{center}
\GapDepth=20pt
\GapWidth=40pt
\begin{bundle}{{\it s-expression}}
\chunk
{
\begin{bundle}{{\it s-expression}}
\chunk
{
\begin{bundle}{{\it variable}}
\chunk{$x_1$}
\end{bundle}
}
\end{bundle}
}
\chunk
{
\begin{bundle}{{\it s-expression}}
\chunk
{
\begin{bundle}{{\it variable}}
\chunk{$x_2$}
\end{bundle}
}
\end{bundle}
}
\end{bundle}
\end{center}


\section{パーサコンビネータ}

パーサコンビネータは,文脈自由文法をそのままに書けるパーサだと思えばよろしい.
パーサコンビネータの実装には,Haskellでは,
Parsec\footnote{\url{http://hackage.haskell.org/package/parsec}},
Attoparsec\footnote{\url{http://hackage.haskell.org/package/attoparsec}},
Trifecta\footnote{\url{http://hackage.haskell.org/package/trifecta}}
などがある.たとえば,冒頭で述べたJavaのif-else文を,HaskellのParsecであらわすと,
次のようになる.
\lstset{basicstyle=\ttfamily,}
\begin{lstlisting}
import qualified Text.Parsec as P

expr :: Monad m => P.ParsecT String u m ()
expr = return ()

stmt :: Monad m => P.ParsecT String u m ()
stmt = P.string "if" >>
        P.char '(' >> expr >> P.char ')' >>
        stmt >> P.string "else" >> stmt
\end{lstlisting}
キーワード{\bf if}は{\tt P.string "if"}のようにして
あらわす.カッコは{\tt P.char '('}のようにしてあらわす.
変数はHaskellにおける識別子であらわす.そして,
それぞれの記号を``\verb|>|\verb|>|"で繋ぐ.
記号を置き換えて読んでみれば,このコードが生成規則をそのまま写したものであるように
見えるのはわかるであろう.

\subsection{正規表現とパーサコンビネータ}

もし正規表現がわかるなら,正規表現を翻訳してみるのがパーサコンビネータを理解する
早道である.たとえば,正規表現{\tt /abc/}を考えてみよう.これは,正規表現の
文法にしたがえば,abcという文字列にマッチする.これをパーサコンビネータに
翻訳すると,{\tt string "abc"}となる.正規表現というならば,当然{\tt *}や{\tt +}
といったオペレータをもつのではないかと疑問に思うだろう.たとえば,{\tt /a*/}はどのように
翻訳されるのか? 答えは,{\tt many (string "a")}である.簡単な対応表を以下に示す.
ここで$\alpha \Longrightarrow \beta$は``$\alpha$は$\beta$に翻訳される"と読む.
ただし,正規表現はPerlのものを意図している.

\begin{itemize}
\item {\tt /x/} $\Longrightarrow$ {\tt string "x"}
\item {\tt /x*/} $\Longrightarrow$ {\tt many (string "x")}
\item {\tt /x+/} $\Longrightarrow$ {\tt many1 (string "x")} または {\tt some (string "x")} (実装によって異なることに注意する.)
\item {\tt /x?/} $\Longrightarrow$ {\tt optional (string "x")}
\item {\tt /x(?R)|\$/} $\Longrightarrow$ {\tt let r = do \{ string "x"; r \verb|<|\verb|||\verb|>| string "$\backslash$n" \}}
\end{itemize}

\section{付録}

簡単なS式パーサの完全な実装をParsec,Attoparsec,Trifectaそれぞれで例として示す.
例のコードはすべて同じ仕様である.例のコードの仕様は,引数なしでコマンドラインから
起動されると,まず1行入力を受け付ける.そしてその行をS式としてパースし,
問題なくパースできたか失敗したかを人間が目で見て判断できるような結果を
出力する.

たとえば,Trifectaの例は,1行目に次のように与えると,
\begin{lstlisting}
(x1.x2)
\end{lstlisting}
次のように出力する.これは抽象構文木を文字の列として表現したものである.
\begin{lstlisting}
Cons (Sym "x1") (Sym "x2")
\end{lstlisting}
1行目に次のように与えると,
\begin{lstlisting}
(x1.x2
\end{lstlisting}
次のように出力する.これはエラーメッセージである.
\begin{lstlisting}
(interactive):1:7: error: unexpected EOF, expected: ")",
    white space
(x1.x2<EOF>
      ^
\end{lstlisting}

付録のサンプルコードをビルドするには,まず本書のソースコードを手に入れて,
本書のソースコードがあるディレクトリまでいく.そしてそのあと,
cabal-devを使用してビルドできる../cabal-dev/bin以下に
バイナリが用意される.
\lstset{basicstyle=\ttfamily,}
\begin{lstlisting}
git clone https://github.com/pasberth/pasberth.github.io.git
cd pasberth.github.io
git checkout default
cd papers/brief-intro-parser-combinators
cabal-dev install
ls ./cabal-dev/bin
\end{lstlisting}

\subsection{Parsec}

Parsecはパーサコンビネータの実装のひとつで,Haskellで書かれている.
以下に,ParsecでS式をパースする完全な例を示す\footnote{
ファイルは
\url{https://raw.github.com/pasberth/pasberth.github.io/default/papers/brief-intro-parser-combinators/sexp-parsec.hs}
から入手できる.}.
\lstset{basicstyle=\ttfamily,}
\begin{lstlisting}
import qualified Text.Parsec as P

data SExp
  = Sym String
  | Cons SExp SExp
  deriving (Show)

sym :: Monad m => P.ParsecT String u m SExp
sym = do
  s <- P.many1 $ P.noneOf "(.) \t"
  return $ Sym s

cons :: Monad m => P.ParsecT String u m SExp
cons = do
  P.char '('
  P.spaces
  x <- sexp
  P.spaces
  P.char '.'
  P.spaces
  y <- sexp
  P.spaces
  P.char ')'
  return $ Cons x y

sexp :: Monad m => P.ParsecT String u m SExp
sexp = cons P.<|> sym

main :: IO ()
main = do
  s <- getLine
  let result
        = P.parse
            (do { x <- sexp ; P.eof ; return x })
            "<stdin>"
            s
  print result
\end{lstlisting}

\subsection{Attoparsec}

Attoparsecはパーサコンビネータの実装のひとつで,Haskellで書かれている.
ByteStringを使用するのでParsecより高速である.
以下に,AttoparsecでS式をパースする完全な例を示す\footnote{
ファイルは
\url{https://raw.github.com/pasberth/pasberth.github.io/default/papers/brief-intro-parser-combinators/sexp-attoparsec.hs}
から入手できる.}.
\lstset{basicstyle=\ttfamily,}
\begin{lstlisting}
{-# LANGUAGE OverloadedStrings #-}

import qualified Control.Applicative as A
import qualified Data.ByteString as B
import qualified Data.Attoparsec as P
import qualified Data.Attoparsec.ByteString.Char8 as P8

data SExp
  = Sym String
  | Cons SExp SExp
  deriving (Show)

sym :: P.Parser SExp
sym = do
  s <- P.many1 $ P8.satisfy (not . (`elem` "(.) \t"))
  return $ Sym s

cons :: P.Parser SExp
cons = do
  P8.char '('
  P.many' P8.space
  x <- sexp
  P.many' P8.space
  P8.char '.'
  P.many' P8.space
  y <- sexp
  P.many' P8.space
  P8.char ')'
  return $ Cons x y

sexp :: P.Parser SExp
sexp = cons A.<|> sym

main :: IO ()
main = do
  s <- B.getLine
  let result
        = P.parse sexp s
  print $ P.eitherResult result
\end{lstlisting}

\subsection{Trifecta}

Trifectaはパーサコンビネータの実装のひとつで,Haskellで書かれている.
今回紹介した3つのなかではいちばん新しく,高度に抽象化されているので,今イチバン
イケてる実装である.
以下に,TrifectaでS式をパースする完全な例を示す\footnote{
ファイルは
\url{https://raw.github.com/pasberth/pasberth.github.io/default/papers/brief-intro-parser-combinators/sexp-trifecta.hs}
から入手できる.}.
\lstset{basicstyle=\ttfamily,}
\begin{lstlisting}
import qualified Control.Applicative as A
import qualified Text.Trifecta as P

data SExp
  = Sym String
  | Cons SExp SExp
  deriving (Show)

sym :: P.Parser SExp
sym = do
  s <- A.some $ P.noneOf "(.) \t"
  return $ Sym s

cons :: P.Parser SExp
cons = do
  P.char '('
  P.spaces
  x <- sexp
  P.spaces
  P.char '.'
  P.spaces
  y <- sexp
  P.spaces
  P.char ')'
  return $ Cons x y

sexp :: P.Parser SExp
sexp = cons A.<|> sym

main :: IO ()
main = do
  s <- getLine
  P.parseTest (do { x <- sexp ; P.eof ; return x }) s
\end{lstlisting}

\subsection{おまけ: Applicative}

Control.ApplicativeはHaskellの標準ライブラリである.
これを使用すると,パーサコンビネータをよりスマートに利用できる.
前の節で書いたTrifectaを使用したコードをApplicativeを使用して書き直す例を示す\footnote{
ファイルは
\url{https://raw.github.com/pasberth/pasberth.github.io/default/papers/brief-intro-parser-combinators/sexp-trifecta-applicative.hs}
から入手できる.}.
\begin{lstlisting}
import           Control.Applicative
import qualified Text.Trifecta as P

data SExp
  = Sym String
  | Cons SExp SExp
  deriving (Show)

sym :: P.Parser SExp
sym = Sym <$> (some $ P.noneOf "(.) \t")

cons :: P.Parser SExp
cons = Cons <$> (P.char '(' *> P.spaces *> sexp)
            <* P.spaces <* P.char '.'
            <*> sexp <* P.spaces <* P.char ')'

sexp :: P.Parser SExp
sexp = cons <|> sym

main :: IO ()
main = do
  s <- getLine
  P.parseTest (sexp <* P.eof) s
\end{lstlisting}

\begin{thebibliography}{数字}
\bibitem{dragonbook} コンパイラ―原理・技法・ツール
\end{thebibliography}

\end{document}